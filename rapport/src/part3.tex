\subsection{Procédures stockées}
\paragraph{Ajout d'article}{On souhaite simplifier la possibilité d'ajouter un article pour les utilisateurs : ceux-ci ne connaissent pas forcément leur identifiant dans la base de données et certains champs peuvent être remplis automatiquement, tel que celui de date de rédaction.
...
}

\paragraph{Contenu d'article}{La plupart du temps, un contenu ajouté à la base de données est utilisé dans un article, on a donc créé une procédure qui permet d'ajouter un contenu à un article. Cette procédure a double effet : d'une part ajouter le nouveau contenu à la table et d'autre part associer ce contenu à l'article dans la table d'association
...
}

\subsection{Vues de la base de données}
\paragraph{}{Pour faciliter la visualisation des données dans notre base, nous avons choisit  de mettre en place quelques vues. Elles nous permettront d'avoir accès rapidement à certaines informations sans avoir à utiliser des requêtes complexes.}

\paragraph{Vues liées aux métiers}{La table \verb Metiers  contient une entrée pour chaque métier existant, et est liée par une clé étrangère à la table Personnes, pour associer une personne à son métier. Il est donc intéressant d'avoir accès rapidement à toutes personnes ayant le même métier. 

Pour ce faire, nous créons une vue par métier recensé dans la table \verb Metiers . Chacun d'entre elle contiendra une requête sur la table \verb Personnes  qui sélectionnera les personnes ayant un id\_metier correspondant à l'id du métier géré par la vue. Par exemple, la vue \verb Illustrateurs  sélectionne les personnes ayant un id\_metier égal à six, car c'est l'identifiant associé au métier d'illustrateur.

Toutes ces vues sont modifiables, et cela pour plusieurs raisons. En effet, elles ne sélectionnent que des attributs NOT NULL et elles n'opèrent que sur une seule table qui n'est pas utilisé dans une sous-requête.}

\subsection{Traitements automatiques}