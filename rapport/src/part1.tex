\subsection{Répartition des tâches}

\paragraph{}{
    Le projet étant réalisé en \textit{trinôme}, il est alors indispensable de répartir au mieux les tâches.
    Le tableau à la figure \ref{planning} présente la répartition des tâches dans le groupe ainsi que les dates prévisionnelles
    de leur réalisation.
}

\begin{figure}
\centering
    \begin{tabular}{|c|p{16em}|p{12em}|}
    	\hline \textbf{semaine} & \textbf{Tâche} & \textbf{Responsable} \\ 
    	\hline S4 & Big table & groupe complet \\ 
    	\hline S5 & Rédaction rapport & groupe complet \\ 
    	\hline S5 & Algo de décomposition & Théo DD \\ 
    	\hline S5 & Algo de Bernstein & Thomas M \\ 
    	\hline S5 & Planning & Benjamin S \\ 
    	\hline S6 & Implémentation SQL des tables & groupe complet \\
    	\hline S6 & Insertion données dans BD & groupe complet \\ 
    	\hline S7 & Définissions des requêtes stockées & Théo DD \\ 
    	\hline S8 & Procédures PL/SQL & Thomas M, Benjamin S \\ 
    	\hline S8 & Mise à jour du planning & Groupe complet \\ 
    	\hline 
    \end{tabular} 
\caption{Répartition des tâches dans le groupe}
\label{planning}
\end{figure}

\paragraph{Remarque}{
    Ce planning n'est ni complet ni exhaustif. En effet, on ne peut pas faire de spéculations sur ce qu'on va mettre dans notre base de données avant d'avoir eu les cours en relation.
}

\subsection{Définition de la \textit{big table}}

\paragraph{}{
    Notre base de données va représenter la gestion des numéros d'un journal. Chaque \textbf{édition} a un numéro, une Une, un prix, une date de parution, un type, un rédacteur en chef. Chaque \textbf{personne} a un numéro d'identification, un nom, un prénom et un numéro de téléphone. Chaque personne a un métier (photographe, rédacteur, etc.) ce qui définit un salaire de base. Un \textbf{contenu} a un titre, un type et un auteur. Dans la base de données on stockera l'emplacement des contenus sous forme d'URL. Un \textbf{article} a une date de rédaction, un titre et un résumé. Un \textbf{article} est composé d'un ensemble de \textbf{contenus} et une \textbf{édition} est composée d'un ensemble d'articles. \newline
    Le schéma à la figure \ref{sch_df} présente toutes les dépendances fonctionnelles de notre base de données.
}
