\paragraph{}{
    Dans le cadre du cours \textit{Base de données 2}, il nous a été demandé de réaliser, par groupe de trois, une base de données réaliste et réalisable en entreprise. Nous avons choisi de modéliser la rédaction d'un journal. Ensuite tout au long du module, nous devons ajouter dans notre base de données des applications des concepts vu en cours.
}

\paragraph{}{
    Dans une première partie, nous détaillerons l'organisation de notre \textit{trinôme} pour la réalisation de ce projet. Ensuite, nous présenterons la conception de notre base de données en partant d'une \textit{Big table}. Nous déterminerons les différentes dépendances fonctionnelles, puis nous appliquerons deux algorithmes de normalisation sur notre table : l'algorithme de Bernstein et l'algorithme de décomposition, tous deux vus en cours.
}

\paragraph{}{
    Nous présenterons également les différentes vues que nous avons mises en place pour visualiser notre base. Nous présenterons également les différents \textit{triggers} créés ainsi que les fonctions stockées pour interagir avec la base de données. \newline
    Enfin, nous discuterons des différents atouts et modifications possibles de notre base de données.
}