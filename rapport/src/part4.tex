%% Critique de la base données
\subsection{Droits d'accès à la base de données}

\paragraph{}{
    Comme nous travaillons à trois sur la base de données, il a fallut que les autres membres de l'équipe puissent voir et avoir accès à ce qu'a fait un autre membre. Pour cela, on a jouer avec  les droits d'accès de la base de données Oracle SQL. Pour simplifier la gestion des droits, on a créé toutes les tables sur un même compte, ce dernier attribuera ensuite des rôles précis à d'autres personnes.
}
\paragraph{}{
    Typiquement pour notre projet, Nous avons deux rôles. Le rôle \verb|resp| et \verb|reader|. Un responsable (\verb|resp|) a tous les droits sur les tables du projet alors qu'un lecteur (\verb|reader|) n'a que le droit de consulter les tables. Le code PL/SQL ci-dessous donner un exemple d'accord de droits sur la table \verb|articles| des droits de sélection, mise à jour, insertion et suppression des données pour le rôle \verb|resp| au utilisateurs L3\_3 et L3\_13. L'utilisateur L3\_4 n'a que le droit de consulter la table \verb|articles|.
}
\begin{lstlisting}[frame=single]
CREATE ROLE resp;
CREATE ROLE reader;

GRANT ALL ON articles TO resp;

GRANT SELECT ON articles TO reader;

GRANT resp TO L3_3, L3_13;
GRANT reader TO L3_4;
\end{lstlisting}

\end{document}

\paragraph{}{
    On procède de même pour chaque table, à savoir \verb|contenus|, \verb|metiers|, \verb|numeros|, \verb|personnes| et \verb|typeJournal|.
}
%Pour pouvoir accéder aux différentes tables, procédures, etc. créé par une pers

\subsection{Atouts}

\subsection{Faiblesses et améliorations}